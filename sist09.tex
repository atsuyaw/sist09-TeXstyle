\documentclass[uplatex]{jsarticle}

\usepackage{sist09}
\usepackage{booktabs}%表組み
\usepackage{mathtools}\mathtoolsset{showonlyrefs=true}%数式番号
\usepackage{gitinfo2}

\usepackage{listings,jlisting}
\lstset{
language={TeX},
%backgroundcolor={\color[gray]{.85}},
basicstyle={\ttfamily\small},
identifierstyle={\small},
commentstyle={\small\ttfamily \color[rgb]{1,0,0}},
keywordstyle={\small\bfseries \color[rgb]{0,0,1}},
ndkeywordstyle={\small\bfseries \color[rgb]{0,1,0}},
stringstyle={\small\ttfamily \color[rgb]{0,0,1}},
frame={tb},
breaklines=true,
columns=[l]{fullflexible},
numbers=left,
xrightmargin=0zw,
xleftmargin=3zw,
numberstyle={\scriptsize},
stepnumber=1,
numbersep=1zw,
morecomment=[l]{//}
}

\author{渡邉充哉}
\date{\today}
\title{sist09.sty}

%titlepageここから
\renewcommand{\sistTitle}{sist09}%標題
\renewcommand{\Engtitle}{sist09}%英文標題
\renewcommand{\reportnumber}{}%レポート番号
\renewcommand{\thesis}{パッケージ説明文書}%レポート名
\renewcommand{\edition}{(v1.0.0)}%レポートの種類・版表示
\renewcommand{\id}{}%学籍番号
\renewcommand{\sistAuthor}{渡邉 充哉}%著者
\renewcommand{\Engauthor}{WATANABE Atsuya}%英文著者名
\renewcommand{\freespace}{}
\renewcommand{\yearr}{2021}%発行年
\renewcommand{\sistDate}{\today}%発行日
\renewcommand{\organization}{~}
\renewcommand{\department}{~}%発行者名
\renewcommand{\Engdepartment}{}%英文発行者名
\renewcommand{\address}{}%発行者所在地
\renewcommand{\tel}{}%発行者電話番号
\renewcommand{\mail}{}
\renewcommand{\editdata}{\sistDate}%版履歴

\begin{document}
\maketitle
% \maketitlepage%表紙
\pagenumbering{roman}
%\makedocumentseet

\begin{abstract}
このスタイルファイルはSIST 09(科学技術情報流通技術基準 科学技術レポートの様式)に沿った文書作成を支援するものです。
\end{abstract}

\tableofcontents%目次
\thispagestyle{empty}%ページ番号消去
\pagenumbering{arabic}
\setcounter{page}{1}

%==========記事ここから==========
\section{変数}
このスタイルファイルでは表\ref{tbl:variable}に示したように多くの変数及び命令を定義した。
\begin{table*}
\centering
\caption{定義した変数及び命令\label{tbl:variable}}
\begin{tabular}{lll}
\toprule
\multicolumn{2}{l}{変数}				&	内容\\
\midrule
\multicolumn{2}{l}{表紙・奥付用変数}\\
&\verb|\sistTitle|		&	表題\\
&\verb|\Engtitle|	&	英文標題\\
&\verb|\reportnumber|	&	レポート番号\\
&\verb|\thesis|		&	レポート名\\
&\verb|\edition|		&	版表示\\
&\verb|\id|			&	学籍番号\\
&\verb|\sistAuthor|		&	著者\\
&\verb|\Engauthor|	&	英文著者名\\
&\verb|\freespace|	&	空白スペース\\
&\verb|\yearr|		&	発行年(数字だけ)\\
&\verb|\sistDate|		&	発行日\\
&\verb|\organization|&	所属機関\\
&\verb|\department|	&	発行者名\\
&\verb|\Engdepartment|&	英文発行者名\\
&\verb|\address|		&	発行者所在地\\
&\verb|\tel|			&	発行者電話番号\\
&\verb|\mail|		&	発行者メールアドレス\\
&\verb|\editdata|	&	版履歴\\
\multicolumn{2}{l}{ドキュメントシート用変数}\\
&\verb|\syoroku|		&	抄録\\
&\verb|\datee|		&	発行日(yyyy-mm-dd)\\
&\verb|\Engaddress|	&	英文発行者所在地\\
&\verb|\ISBN|		&	ISBN\\
&\verb|\ISSN|		&	ISSN\\
&\verb|\keyword|		&	キーワード\\
&\verb|\thesaurus|	&	シソーラス\\
&\verb|\category|	&	分類(NDC等)\\
&\verb|\valuee|		&	価格\\
命令
&\verb|\maketitlepage|		&	表紙の出力\\
&\verb|\makedocumentseet|	&	ドキュメントシートの出力\\
&\verb|\makeokuzuke|		&	奥付の出力\\
\bottomrule
\end{tabular}
\end{table*}

\section{依存関係}
必要なパッケージは以下の通りです。
\begin{itemize}
    \item geometry
    \item booktabs
    \item url
    \item hyperref
    \item pxjahyper
\end{itemize}

\section{使用例}
\verb|sist09templete.tex|をタイプセットすると次のような結果が得られます。
\clearpage
\lstinputlisting[caption=テンプレート ``sist09.tex'',label=templete]{sist09templete.tex}
\clearpage

\renewcommand{\sistTitle}{\tt\textbackslash sistTitle}%標題
\renewcommand{\Engtitle}{\tt\textbackslash Engtitle}%英文標題
\renewcommand{\reportnumber}{\tt\textbackslash reportnumber}%レポート番号
\renewcommand{\thesis}{\tt\textbackslash thesis}%レポート名
\renewcommand{\edition}{\tt\textbackslash edition}%レポートの種類・版表示
\renewcommand{\id}{\tt\textbackslash id}%学籍番号
\renewcommand{\sistAuthor}{\tt \textbackslash sistAuthor}%著者
\renewcommand{\Engauthor}{\tt\textbackslash Engauthor}%英文著者名
\renewcommand{\freespace}{\tt\textbackslash freespace}
\renewcommand{\yearr}{\tt\textbackslash yearr}%発行年
\renewcommand{\sistDate}{\tt\textbackslash sistDate}%発行日
\renewcommand{\organization}{\tt\textbackslash organization}
\renewcommand{\department}{\tt\textbackslash department}%発行者名
\renewcommand{\Engdepartment}{\tt\textbackslash Engdepartment}%英文発行者名
\renewcommand{\address}{\tt\textbackslash address}%発行者所在地
\renewcommand{\tel}{\tt\textbackslash tel}%発行者電話番号
\renewcommand{\mail}{\tt\textbackslash mail}
\renewcommand{\editdata}{\tt\textbackslash editdata}%版履歴
\renewcommand{\syoroku}{\tt\textbackslash syoroku}
\renewcommand{\datee}{\tt\textbackslash datee}
\renewcommand{\Engaddress}{\tt\textbackslash Engaddress}
\renewcommand{\ISBN}{\tt\textbackslash ISBN}
\renewcommand{\ISSN}{\tt\textbackslash ISSN}
\renewcommand{\keyword}{\tt\textbackslash keyword}
\renewcommand{\thesaurus}{\tt\textbackslash thesaurus}
\renewcommand{\category}{\tt\textbackslash category}
\renewcommand{\valuee}{\tt\textbackslash valuee}
\maketitlepage
\makedocumentseet

%\makeokuzuke
%----------奥付----------
%mailリンク内で\textbackslashが使えないので。
\begin{table}[b]
\centering
\begin{tabular}{ll}
\vbox to 180mm{\vfill}	\\
\multicolumn{2}{l}{\small\thesis}	\\
\multicolumn{1}{l}{\Large\sistTitle}&	\\
\toprule
\editdata	\\
%	&編集並びに発行	\\
\sistAuthor\\
\organization	\\
\department	\\
\address	\\
TEL: \tel	\\ 
Mail: \mail	\\
\bottomrule
%\multicolumn{2}{c}{\copyright \yearr \organization}
\end{tabular}
\end{table}
%----------奥付----------
\renewcommand{\sistTitle}{sist09}%標題
\renewcommand{\Engtitle}{sist09}%英文標題
\renewcommand{\reportnumber}{}%レポート番号
\renewcommand{\thesis}{スタイルファイル解説文書}%レポート名
\renewcommand{\edition}{v1.0.0}%レポートの種類・版表示
\renewcommand{\id}{}%学籍番号
\renewcommand{\sistAuthor}{渡邉 充哉}%著者
\renewcommand{\Engauthor}{WATANABE Atsuya}%英文著者名
\renewcommand{\freespace}{}
\renewcommand{\yearr}{2021}%発行年
\renewcommand{\sistDate}{\today}%発行日
\renewcommand{\organization}{~}
\renewcommand{\department}{~}%発行者名
\renewcommand{\Engdepartment}{}%英文発行者名
\renewcommand{\address}{}%発行者所在地
\renewcommand{\tel}{}%発行者電話番号
\renewcommand{\mail}{}
\renewcommand{\editdata}{\sistDate}%版履歴

%==========記事ここまで==========
\end{document}
